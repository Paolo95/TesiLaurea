\chapter{Discussione in merito al lavoro svolto}

\begin{preamble}
{\em
Il capitolo sarà incentrato nell'analisi del lavoro svolto per la creazione dell'applicazione Android. In particolare, si cercherà di analizzare le differenze e i punti in comune con l'applicazione Vivino, che attualmente è molto in voga tra gli utenti in modo tale da mettere in risalto le potenzialità dell'applicazione mobile. \newline \indent Successivamente, verrà mostrata un'approfondita SWOT Analysis che cercherà di analizzare i rischi associati allo sviluppo e nella distribuzione futura dell'applicazione. \newline \indent Infine, si illustreranno gli sviluppi futuri del progetto a partire dalla roadmap relativa allo sviluppo dell'applicazione fino alle possibili implementazioni e feature aggiuntive che potrebbero essere aggiunte in futuro per migliorare le possibilità offerte dall'applicazione. 
}
\end{preamble}


\section{Confronto con l'app Vivino}



\section{SWOT Analysis del progetto}

L'analisi SWOT (\textit{Strengths}, \textit{Weaknesses}, \textit{Opportunities}, \textit{Threats}) è uno strumento di pianificazione e valutazione dei rischi utilizzato nelle aziende, nelle organizzazioni, e nella pianificazione personale. Consiste nell'identificazione e nell'analisi dei punti di forza, delle debolezze, delle opportunità e delle minacce rilevanti in una determinata situazione o contesto. I fattori chiave dell'analisi SWOT sono:

\begin{itemize}

	\item Punti di forza \textit{(Strengths)}: questi sono gli aspetti interni positivi, risorse e capacità che un'organizzazione o un individuo possiede. I punti di forza possono includere competenze specializzate, buone relazioni con i clienti, un marchio forte, risorse finanziarie solide, o qualsiasi altro vantaggio competitivo.
	
	\item Debolezze \textit{(Weaknesses)}: le debolezze sono gli aspetti interni negativi o le carenze che limitano l'efficacia. Questi possono includere una gestione inefficace, carenze nelle competenze del personale, mancanza di risorse finanziarie o problemi operativi.

	\item Opportunità \textit{(Opportunities)}: le opportunità sono situazioni esterne positive o tendenze di mercato che possono essere sfruttate per ottenere vantaggi. Queste possono includere l'emergere di nuovi mercati, cambiamenti normativi favorevoli, nuove tecnologie o altre circostanze favorevoli.
	
	\item Minacce \textit{(Threats)}: le minacce sono situazioni esterne negative o fattori che possono rappresentare una sfida o un rischio. Questi possono includere la concorrenza intensa, l'instabilità economica, cambiamenti normativi sfavorevoli o altre minacce al successo.

\end{itemize}

Per condurre una SWOT analysis, è importante identificare con precisione questi fattori, sia interni che esterni. Gli elementi interni sono sotto il controllo dell'organizzazione e possono essere influenzati direttamente, mentre gli elementi esterni sono fuori dal controllo dell'organizzazione ma possono comunque avere un impatto significativo sulla sua strategia.

Una SWOT analysis aiuta da identificare come sfruttare i punti di forza e le opportunità, mitigare i punti deboli e affrontare le minacce. Questa analisi fornisce una base solida per la pianificazione strategica e la presa di decisioni aziendali informate.

Di seguito, per ogni rischio associato allo sviluppo e messa in produzione dell'applicazione ne verrà mostrata la \textit{SWOT Analysis}. 

\subsection{Difficoltà nella diffusione dell'app tra gli utenti}

\definecolor{CreamBrulee}{rgb}{1,0.898,0.6}
\definecolor{BarleyWhite}{rgb}{1,0.949,0.8}
\definecolor{DoublePearlLusta}{rgb}{0.988,0.898,0.803}
\begin{table}
\centering
\begin{tblr}{
  row{1} = {CreamBrulee},
  row{2} = {DoublePearlLusta},
  column{2} = {BarleyWhite},
  cell{1}{1} = {c=2}{},
  cell{1}{3} = {c=2}{},
  cell{2}{1} = {r=3}{CreamBrulee},
  vlines,
  hline{1-2,5} = {-}{},
  hline{3-4} = {2-4}{},
}
                          &         & CARATTERE DEL FATTORE                                                                                                         &                                                                       \\
{ORIGINE DEL \\FATTORE  } &         & POSITIVO                                                                                                                      & NEGATIVO                                                              \\
                          & INTERNO & {STRENGHTS\\\labelitemi\hspace{\dimexpr\labelsep+0.5\tabcolsep}test\\\labelitemi\hspace{\dimexpr\labelsep+0.5\tabcolsep}ciao} & {WEAKNESSES\\\labelitemi\hspace{\dimexpr\labelsep+0.5\tabcolsep}test} \\
                          & ESTERNO & {OPPORTUNITIES\\\labelitemi\hspace{\dimexpr\labelsep+0.5\tabcolsep}test}                                                      & {THREADS\\\labelitemi\hspace{\dimexpr\labelsep+0.5\tabcolsep}test}    
\end{tblr}
\end{table}

\section{Sviluppi futuri del progetto}


