\chapter{Discussione in merito al lavoro svolto}

\begin{preamble}
{\em
Questo capitolo sarà incentrato sull'analisi del lavoro dedicato alla creazione dell'applicazione Android. In particolare, saranno esaminate le differenze e le similitudini con l'applicazione Vivino, attualmente molto popolare tra gli utenti, al fine di mettere in evidenza le potenzialità dell'applicazione mobile oggetto della tesi. \newline \indent Successivamente, verrà condotta un'approfondita analisi SWOT per evidenziare i punti di forza, di debolezza, le opportunità e le minacce della nostra applicazione.
}
\end{preamble}

\section{Confronto con l'app Vivino}

Vivino (Figura \ref{7fig:logoVivino}) è una popolare app mobile e una piattaforma online che si concentra sulla degustazione e sulla condivisione di recensioni e informazioni sul vino. Gli utenti di Vivino possono utilizzare l'app per scansionare etichette di bottiglie di vino e ottenerne informazioni dettagliate, come recensioni, punteggi, descrizioni, abbinamenti cibo-vino e prezzi. Inoltre, gli utenti possono anche recensire i vini, tenere traccia dei loro preferiti e connettersi con altri amanti del vino sulla piattaforma.

Vivino è diventato uno strumento popolare per gli appassionati di vino che desiderano esplorare e scoprire nuovi prodotti, condividere le proprie esperienze e prendere decisioni informate sugli acquisti. La piattaforma contiene una vasta quantità di dati sul vino e gestisce una comunità di appassionati del settore in cui è possibile condividere opinioni e consigli.

L'applicazione oggetto della presente tesi condivide molti aspetti in comune con l'applicazione Vivino, poiché entrambi sfruttano il riconoscimento delle bottiglie di vino attraverso l'utilizzo della fotocamera. Tuttavia, le differenze risiedono negli obiettivi in quanto Vivino si concentra sulla creazione di una community di appassionati del vino che condividono recensioni e descrizioni sui prodotti vitivinicoli. D'altra parte, l'applicazione oggetto della presente tesi combina le informazioni di base sul vino con lo stato di salute del vigneto di provenienza, utilizzando sia i dati satellitari che quelli provenienti dalla stazione IoT.

\begin{figure}[h]
	\centering
	\includegraphics [width=.45\columnwidth, angle=0]
            {logoVivino}
	\caption{Logo dell'applicazione Vivino}
	\label{7fig:logoVivino}
\end{figure}

\section{SWOT Analysis del progetto}

L'analisi SWOT (\textit{Strengths}, \textit{Weaknesses}, \textit{Opportunities}, \textit{Threats}) è uno strumento di pianificazione e valutazione utilizzato nelle aziende, nelle organizzazioni, e persino dalle singole persone. Consiste nell'identificazione e nell'analisi dei punti di forza, di debolezza, delle opportunità e delle minacce rilevanti in una determinata situazione o contesto. I fattori chiave dell'analisi SWOT sono:

\begin{itemize}

	\item Punti di forza \textit{(Strengths)}: identifica i punti di forza interni che possono mitigare i rischi.
	
	\item Debolezze \textit{(Weaknesses)}: identifica le debolezze interne che possono aumentare i rischi.

	\item Opportunità \textit{(Opportunities)}: identifica le opportunità esterne che possono ridurre i rischi o fornire alternative.
	
	\item Minacce \textit{(Threats)}: identifica le minacce esterne che possono aumentare i rischi.

\end{itemize}

Per condurre una \textit{SWOT analysis} è importante identificare con precisione questi fattori, sia interni che esterni. Gli elementi interni sono sotto il controllo dell'organizzazione e possono essere influenzati direttamente, mentre gli elementi esterni sono fuori dal controllo dell'organizzazione, ma possono comunque avere un impatto significativo sulla strategia.

Una \textit{SWOT analysis} aiuta ad identificare il modo in cui sfruttare i punti di forza e le opportunità, mitigare i punti deboli e affrontare le minacce. Questa analisi fornisce una base solida per la pianificazione strategica e la presa di decisioni aziendali informate.

Per condurre la \textit{SWOT Analysis}, è importante definire il concetto di \textit{Smoke Test}. Lo \textit{Smoke Test} rappresenta una metodologia essenziale per la validazione di un'idea di business, garantendo che il prodotto o servizio proposto sia realmente in linea con le aspettative dei futuri clienti. Lo scopo principale di questo test è determinare se i potenziali clienti manifestano un interesse reale per l'offerta. La chiave del successo è far percepire loro che il prodotto o servizio è già disponibile sul mercato. Questo processo è cruciale per validare l'idea e coinvolgere potenziali clienti ancor prima del lancio ufficiale del prodotto.

La \textit{SWOT Analysis} dell'applicazione è riportata di seguito.

\begin{itemize}
	\item Strenghts:
	\begin{itemize}
		\item L'applicazione fornisce in tempo reale informazioni sullo stato di salute del vigneto, facilitando il lavoro degli operatori agricoli nella gestione dei terreni coltivati. Gli utenti possono ottenere dettagli sul vino di interesse, aumentando la loro consapevolezza riguardo al prodotto vinicolo che stanno acquistando.
		\item L'azienda potrebbe ampliare le proprie operazioni estendendo l'analisi dei dati non solo al vino, ma anche ad altri prodotti vitivinicoli. Questo approccio potrebbe incrementare le opportunità di diffusione dell'applicazione.
		\item L'applicazione risulta molto intuitiva nell'utilizzo e questo migliora notevolmente il tempo di permanenza dell'utente e il suo grado di soddisfazione durante l'utilizzo dell'app.
		\item L'applicazione è totalmente gratuita per gli utenti, massimizzandone le probabilità di diffusione.	
	\end{itemize}
	\item Weaknesses:
	\begin{itemize}
		\item Il team di sviluppo software dovrà essere ampliato in seguito all'introduzione di nuove feature, aumentando, così, il costo di sviluppo dell'applicazione.
		\item Allo stato attuale, l'applicazione non è scaricabile dalla piattaforma \textit{Google Play Store}. Questo aspetto può aumentare le difficoltà nell'eseguire lo \textit{Smoke Test}.
	\end{itemize}
	\item Opportunities:
	\begin{itemize}
		\item L'azienda \textit{Trace Technologies} mantiene un costante dialogo con le aziende vitivinicole tramite il servizio "Vigneto Sicuro", raccogliendo feedback in modo continuo per individuare nuove esigenze nel settore agricolo. Inoltre, l'azienda si impegna a soddisfare tali necessità attraverso l'efficiente utilizzo dei dati in agricoltura, integrando le soluzioni richieste direttamente nell'applicazione.
		\item L'applicazione è in costante evoluzione e può adattarsi rapidamente alle esigenze degli utenti, identificate attraverso lo \textit{Smoke Test}. Questo permette di migliorare rapidamente la diffusione dell'applicazione, garantendo un allineamento tempestivo con le richieste degli utenti.	
		\item I settori in cui l'applicazione è inserita abbracciano due ambiti fortemente in crescita, come l'analisi dei dati e la viticoltura.
	\end{itemize}
	\item Threats:
	\begin{itemize}
		\item Lo \textit{Smoke Test} potrebbe rivelare nuove funzionalità non previste ma molto richieste dagli utenti, generando un aumento dell'impegno da parte degli sviluppatori e, di conseguenza, un aumento dei costi nello sviluppo dell'applicazione.
		\item Gli utenti mostrano una bassa retention durante l'utilizzo dell'applicazione, probabilmente a causa di un'esperienza utente non sufficientemente coinvolgente o di informazioni presenti che non risultano di interesse per l'utilizzatore finale.
	\end{itemize}
\end{itemize}