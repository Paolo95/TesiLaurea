\chapter{Confronto con sistemi di tracciamento}

\begin{preamble}
{\em
In questo capitolo, si tratteranno le differenze sostanziali tra i sistemi di tracciamento introdotti nei capitoli iniziali e l'applicazione Android che è alla base del progetto illustrato nella tesi. \newline \indent La prima sezione sarà dedicata all'analisi degli aspetti in comune dell'applicazione con i sistemi di tracciamento analizzati in precedenza, con un approfondimento sulle possibilità offerte dal progetto nel complesso sistema del tracciamento e richiamo dei prodotti alimentari che presentano delle anomalie pericolose per la salute umana. \newline \indent La seconda parte, invece, cercherà di esaminare le differenze che intercorrono tra i sistemi di tracciamento in generale e l'applicazione Android.
}
\end{preamble}

\section{Aspetti in comune del progetto con i sistemi di tracciamento}

Come precedentemente menzionato, i sistemi di tracciamento rivestono un ruolo cruciale in diverse industrie, tra cui quella alimentare, garantendo la sicurezza e l'affidabilità dei prodotti sul mercato.

L'applicazione Android oggetto di questa tesi presenta alcuni aspetti in comune con i sistemi di tracciamento impiegati nell'industria alimentare. L'app permette ai consumatori di monitorare lo stato di salute di un vigneto, utilizzando dati in tempo reale provenienti dall'API OpenWeather e da una stazione IoT situata vicino al vigneto. In futuro, potrebbe essere arricchita con un sistema di previsione della salute del vigneto, basato sull'analisi storica dei dati, consentendo un tracciamento accurato dello stato del vigneto nel tempo. Un'analisi degli sviluppi futuri verrà effettuata nel capitolo successivo.

L'azienda Trace Technologies sta sviluppando "Vigneto Sicuro", una soluzione proprietaria in grado di prevedere alcune malattie del vigneto, come, ad esempio, la peronospora. Nel prossimo capitolo, si esplorerà la possibilità di integrare l'intero sistema all'interno dell'applicazione, offrendo un avanzato strumento di tracciamento ai consumatori. Questo può avere due obiettivi principali; il primo è consentire all'applicazione di diventare uno strumento utile nelle operazioni di tracciamento della filiera vitivinicola e il secondo è quello di fornire una migliore validazione e certificazione della qualità del vino.

Come discusso nel secondo capitolo, i sistemi di tracciamento sono essenziali per effettuare richiami in caso di criticità nella filiera produttiva, come la presenza eccessiva di pesticidi o sostanze dannose per la salute. L'applicazione può essere utilizzata efficacemente per rilevare la presenza eccessiva di sostanze pericolose nel terreno impiegato nella produzione di un particolare prodotto vitivinicolo. Con una rete più ampia di sensori nel campo, si potrebbe segnalare in modo tempestivo e preciso un aumento anomalo della concentrazione di pesticidi nel suolo, identificandone il punto esatto nel vigneto.

Queste informazioni sono cruciali nel momento in cui le autorità sanitarie conducono le attività di tracciamento nella filiera produttiva, consentendo loro di agire più rapidamente. In alcuni casi, i dati ottenuti dall'applicazione potrebbero persino innescare l'avvio di attività di tracciamento dei prodotti alimentari. L'applicazione, quindi, rappresenta un prezioso strumento di rilevazione e monitoraggio delle anomalie nella filiera produttiva, collaborando con le autorità sanitarie per supportare le operazioni di tracciamento e richiamo.

\section{Aspetti differenti del progetto con i sistemi di tracciamento}

L'applicazione mobile non costituisce un sistema di tracciamento completo, poiché il progetto non prevede la sorveglianza dell'intera filiera di un prodotto alimentare. Tuttavia, essa svolge un ruolo rilevante nella raccolta, catalogazione e condivisione di dati in collaborazione con le autorità competenti durante una campagna di richiamo, se necessario. 

Il sistema non fornisce gli strumenti necessari per condurre indagini approfondite che riguardano tutti gli attori coinvolti nella filiera produttiva di un lotto sospetto. L'applicazione è in grado di condividere i dati relativi alle aziende legate alla produzione di prodotti vitivinicoli, ma per effettuare richiami su larga scala si richiedono molte altre informazioni e risorse non presenti nell'applicazione mobile.

Infatti, il suo scopo principale è la visualizzazione all'utente di informazioni dettagliate sul prodotto vitivinicolo, con l'obiettivo di educare e sensibilizzare il consumatore sull'enologia in generale. I concetti esposti nella sezione precedente rappresentano solo alcuni degli utilizzi potenziali dell'applicazione, dato che essa può contenere in futuro dati di interesse per le autorità sanitarie incaricate nell'effettuare operazioni di rintracciamento e richiamo di prodotti alimentari soggetti a indagine.

Un altro aspetto di rilievo che differenzia questa applicazione da un sistema di tracciamento completo è il suo obiettivo principale, che non consiste nella segnalazione di criticità nella filiera produttiva alle autorità competenti. Infatti, è importante sottolineare che l'applicazione offre un quadro di monitoraggio e raccolta dati, ma non è dotata di strumenti per condurre approfondite indagini o ispezioni su eventuali irregolarità nell'intera filiera produttiva. Pertanto, per tali attività, sarebbe necessario coinvolgere altre risorse e competenze al di fuori del suo ambito d'applicazione.