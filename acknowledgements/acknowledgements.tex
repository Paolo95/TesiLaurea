\chapter*{Ringraziamenti}
\addcontentsline{toc}{chapter}{Ringraziamenti}
\markboth{Ringraziamenti}
\emph{

Dopo la laurea triennale, il mio obiettivo era uscire dalla mia comfort zone e affrontare nuove sfide. Inizialmente convinto di non aver bisogno della laurea magistrale, ho poi capito che dovevo completare il mio percorso di crescita. Come afferma Benazir Bhutto, "Una nave in porto è al sicuro, ma non è per questo che le navi sono state costruite". Volevo essere la nave che lascia il porto, solo che all'inizio avevo sbagliato rotta perché forse era meglio continuare a studiare. Ringrazio coloro che sono stati al mio fianco in questo difficile e faticoso percorso. La mia presenza qui è anche merito vostro.

La prima persona che voglio ringraziare è Valentina. Ti ringrazio infinitamente per aver sopportato gli enormi monologhi fatalisti ed autolesionistici che partivano senza controllo dalla mia mente nei periodi di maggiore stress pre-esame. Grazie per essere la mia bussola nei momenti in cui perdo l'orientamento e credo di non poter superare gli ostacoli che incontro durante il cammino. Sei l'isola in cui rifugiarsi quando fuori il mare è tempestoso.

Ringrazio la mia famiglia, per aver sempre supportato le mie scelte senza mai pormi dei vincoli dandomi conforto nei momenti di difficoltà.

Se sono arrivato qui è grazie ai miei amici fondamentali per staccare la spina nei momenti più intensi. Non riesco veramente a riassumere in poche righe quanto vi sia riconoscente per tutto quello che fate per me. 

Ringrazio infinitamente Marco e Carlo di Trace Technologies. Potevate essere inclusi tra i ringraziamenti agli amici ma qui vi ringrazio anche da un punto di vista professionale. Grazie per aver reso possibile il lavoro di tirocinio. Senza di voi tutto questo lavoro non esisterebbe.

Ovviamente, una delle differenza fondamentali tra la laurea triennale e magistrale è stato il numero maggiore di progetti da portare a termine per ogni esame. Ringrazio tanto i miei colleghi Simone Francalancia, Simone Onori, Nicola e Giacomo per il numero incalcolabile di ore passate in call per cercare di finire in tempo i progetti.

Infine ringrazio, e non in ordine di importanza, il professor Ursino. Se ho ripreso a programmare con il sorriso e con quella spensieratezza che avevo quando ero alle superiori è solo grazie a lei.

Alla fine di questo percorso non sono che all'inizio di questo lungo viaggio che è la vita. Non so minimamente cosa mi aspetterà, ma so per certo che con voi accanto sarà bellissimo lasciare il mio porto sicuro.
}