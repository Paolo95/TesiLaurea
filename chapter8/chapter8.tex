\phantomsection
\addcontentsline{toc}{chapter}{Conclusioni}
\chapter*{Conclusioni}

Il lavoro svolto nella tesi si è concentrato sullo sviluppo del prototipo di un'applicazione Android in grado di riconoscere le bottiglie di vino al fine di visualizzare informazioni relative allo stato di salute del vigneto di provenienza. Inizialmente, sono stati identificati gli strumenti software necessari per l'implementazione dell'applicazione Android, come \textit{Unity}, \textit{Vuforia Engine}, e \textit{Polycam}. Successivamente, sono stati definiti nel dettaglio gli aspetti progettuali del lavoro. Dopo la selezione degli strumenti, sono state scelte le versioni di ciascun software per evitare problemi di compatibilità, a cui ha fatto seguito la progettazione dettagliata dei Class Diagram e Sequence Diagram per ottenere una visione più chiara degli aspetti software cruciali dell'applicazione.

Successivamente, sono state presentate e descritte approfonditamente le classi C\# rilevanti per il progetto per chiarire ogni aspetto implementativo dell'applicazione e giustificare le scelte progettuali effettuate durante la progettazione.

L'ultima parte del lavoro si è concentrata sul confronto dell'applicazione con i sistemi di tracciamento dei prodotti alimentari, con l'obiettivo di analizzare i punti in comune e le differenze tra l'applicazione e questi sistemi.

Successivamente, è stata condotta una discussione dei punti di forza, di debolezza, delle opportunità e minacce dell'applicazione, fornendo un'accurata SWOT Analysis. 

Di seguito, sono indicati i possibili sviluppi futuri del prototipo dell'applicazione mobile:

\begin{enumerate}
	\item Un primo sviluppo, potrebbe riguardare l'implementazione di un sistema per fornire ed elaborare dati rilevanti alle autorità sanitarie nel momento in cui è necessaria una campagna di richiamo di un lotto pericoloso nel settore vitivinicolo.
	\item Un'altra possibile evoluzione dell'applicazione potrebbe coinvolgere l'integrazione della piattaforma web proprietaria di Trace Technologies, denominata "Vigneto Sicuro", all'interno di essa. Ciò consentirebbe agli utenti registrati e agli operatori del settore di accedere ad ulteriori dati su specifici prodotti vitivinicoli.
	\item Potrebbe essere considerata la creazione di una piattaforma per la condivisione delle recensioni da parte dei sommelier. Questi esperti potranno condividere le informazioni sui vini registrati all'interno della piattaforma fornita da Trace Technologies, incoraggiando le aziende agricole a sottoscrivere l'abbonamento alla piattaforma "Vigneto Sicuro".
\end{enumerate}
