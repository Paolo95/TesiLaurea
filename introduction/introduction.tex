\phantomsection
\addcontentsline{toc}{chapter}{Introduzione}
\chapter*{Introduzione}
\markboth{Introduzione}{}

Negli ultimi decenni i dati hanno assunto un'importanza sempre maggiore nella vita di un'azienda e, soprattutto, rappresentano un vantaggio competitivo, se sono ben utilizzati. Per fare ciò, essi devono essere analizzati e studiati a fondo. L’analisi dei dati è un processo di ispezione, pulizia, trasformazione e modellazione
di dati con il fine di evidenziare informazioni utili che supportino le decisioni strategiche aziendali.
Se le decisioni aziendali vengono prese senza l'utilizzo della conoscenza dei dati, non parliamo di una \textit{decisione strategica} ma di una semplice \textit{ipotesi}.

Le aziende sono in possesso di una grande mole di dati; esse, spesso, non riescono ad interpretarli e sfruttare l'enorme potenziale che essi hanno. Per capire questo concetto, riportiamo una frase del matematico francese, Henri Poincaré, ovvero: ``la scienza è fatta di dati come una casa è fatta di pietre. Ma un ammasso di dati non è scienza più di quanto un mucchio di pietre sia una vera casa". Questa frase ribadisce il fatto che, nonostante un'azienda sia in possesso di tantissimi dati, non necessariamente questi portano informazioni utili all'azienda. Tuttavia, attraverso una fase accurata di analisi, possono portare l'azienda stessa al successo. 

I dati hanno un potenziale informativo enorme, che può aiutare le imprese sia a conoscere meglio se stesse (utilizzando i dati interni) sia il proprio mercato (utilizzando dati esterni), ma soprattutto i propri clienti.  Secondo Peter Drucker, considerato il padre del management moderno, lo scopo di un’azienda è quello di creare e mantenere stretto un cliente attraverso il marketing e l’innovazione. \uppercase{è} molto importante soddisfare il cliente con i propri prodotti/servizi;  però, la maggior parte delle volte questo è molto difficile a causa della concorrenza.  Per questo motivo, nell'ultimo  periodo, sono tante le imprese che  hanno  iniziato a  considerare  di  fondamentale  importanza  la  \textit{customer  analytics},  per  conoscere  i propri clienti ed ottenere risultati migliori. Grazie a questa attività si riescono a capire i comportamenti dei propri clienti e le preferenze di questi ultimi al fine di prendere decisioni commerciali strategiche  e  tattiche. 


In generale si può dire che tutti i dati sono importanti per il business, perché la maggior parte aiuta a scoprire e analizzare le aspettative dei clienti e le opportunità di crescita più efficaci. I dati utilizzati per eseguire questo tipo di analisi, la maggior parte delle volte,  possono provenire da diversi reparti di un'azienda e riguardano le vendite, gli acquisti, i prodotti, ma  anche le promozioni realizzate dall'azienda stessa.

Queste ultime svolgono un ruolo fondamentale nel soddisfare e attirare nuovi clienti,  in  quanto  stimolano la  richiesta  del  prodotto/servizio che,  all'occhio del cliente, deve  sembrare più  attraente e innovativo. In base agli obiettivi dell'azienda, le strategie promozionali possono essere di tipo \textit{pull} o \textit{push}. Le prime cercano  di  far  sì  che  i  clienti ``tirino”  i  prodotti  dall'azienda, attraverso, per esempio, l'utilizzo di sconti. Le seconde cercano di spingere il prodotto dall'azienda  verso  distributori  e  rivenditori. Le prime permettono di costruire un legame più forte con il cliente, in quanto è a stretto contatto con questo.     



In questo elaborato sono stati analizzati i dati relativi alle promozioni e alle vendite dell'azienda Fater. In una prima fase viene effettuata una pulizia dei dati; successivamente, viene eseguita un'esplorazione di questi per comprenderli meglio. Ci sarà, anche, un'analisi descrittiva dei contenuti più interessanti all'interno dei dataset. L'obiettivo finale è quello di capire quanto le promozioni, realizzate dall'azienda, hanno influenzato le vendite; per fare ciò, viene eseguita una fase di clustering; successivamente, per riuscire a esprimere quanto un nuovo prodotto in  promozione influenzerà la vendita, viene svolta una fase di classificazione.

La presente tesi è composta da sette capitoli strutturati come di seguito specificato:
\begin{itemize}
	\item Nel Capitolo 1 saranno introdotti il mondo della customer analytics e, successivamente, il ruolo delle promozioni nelle campagne di marketing.
	\item Nel Capitolo 2 si presenteranno le attività relative alle campagne promozionali in Fater. In particolare, si introdurranno le due tipologie di sconti utilizzate e la gerarchia dei clienti.
	\item Nel Capitolo 3 verrà introdotto il sistema dell'azienda riguardante le promozioni; successivamente, saranno illustrati i set di dati relativi alle promozioni e alle vendite dell'azienda.
	\item Nel Capitolo 4 verranno analizzate le fasi di ETL e di EDA sui dataset delle promozioni e delle vendite e sarà effettuata un'analisi descrittiva dei dati.
	\item Nel Capitolo 5 verranno analizzate le attività di clustering e classificazione dei dataset.
	\item Nel Capitolo 6 saranno riportati i risultati ottenuti.
	\item Nel Capitolo 7 verranno tratte le conclusioni e verranno delineati alcuni possibili sviluppi futuri.
	
\end{itemize}
