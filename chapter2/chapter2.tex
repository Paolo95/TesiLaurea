\chapter{Il tracciamento dei prodotti}

\begin{preamble}
{\em 
In questo capitolo si fornirà un'ampia visione della tracciabilità dei prodotti. Si inizierà descrivendo in dettaglio le responsabilità degli operatori del settore e le motivazioni che spingono al continuo sviluppo di questa disciplina. Successivamente, si metterà a confronto il tracciamento dei prodotti con altri settori economici ed, infine, si concluderà il paragrafo con uno sguardo al futuro. \newline \indent In seguito, si esamineranno il tracciamento dei prodotti in diversi ambiti, come la sanità pubblica, l'industria, e la tracciabilità dei prodotti sfusi, inclusa la carne e il pollame. \newline \indent Infine, l'ultima sezione del capitolo sarà dedicata alla blockchain, con un'analisi approfondita del suo utilizzo nella tracciabilità dei prodotti.
}
\end{preamble}

\section{Introduzione}

La \textit{tracciabilità}, in termini generali, si riferisce alla capacità di seguire l'intera storia e i dettagli relativi al movimento dei prodotti. Questo concetto ha diverse applicazioni in vari contesti. La tracciabilità può essere impiegata per attestare la genealogia di individui o oggetti, mentre in alcuni settori è strettamente legata alla gestione della catena del freddo.

Tuttavia, nel contesto della tesi, la tracciabilità e la rintracciabilità dei prodotti si riferiscono specificamente al monitoraggio del percorso dei prodotti alimentari lungo l'intera catena di approvvigionamento. Questo processo comprende diverse fasi, tra cui la coltivazione, il raccolto, la produzione, l'imballaggio, lo stoccaggio, la distribuzione e la vendita ai consumatori finali. Le normative vigenti possono variare nei requisiti di registrazione in ciascuna di queste fasi, ma è importante notare che ogni punto della catena, sia esso un luogo di origine o destinazione del prodotto alimentare, contribuisce al tracciamento e alla rintracciabilità complessiva.

La rintracciabilità può essere focalizzata su due direzioni principali: la \textit{trace forward}, che segue il flusso dei prodotti verso il consumatore finale, e la \textit{trace back}, che risale il percorso dei prodotti fino alla loro origine. Poiché la rintracciabilità è strettamente connessa al movimento fisico dei prodotti, è spesso associata alle dinamiche delle catene di approvvigionamento e alla gestione logistica.

Tuttavia, quando si tratta di discutere di rintracciabilità nell'industria alimentare, il contesto predominante è quello della sicurezza alimentare. La comunità della salute pubblica e le autorità di regolamentazione pongono particolare enfasi sulla sicurezza alimentare, specialmente durante le indagini relative a focolai di malattie alimentari. In questi casi, la rintracciabilità diventa uno strumento cruciale per individuare l'origine del problema e prevenirne la diffusione.

Al contempo, l'industria alimentare considera la sicurezza alimentare come una priorità fondamentale, si impegna principalmente a prevenire eventi di contaminazione, piuttosto che affidarsi esclusivamente alla rintracciabilità come soluzione retroattiva. In altre parole, l'obiettivo principale è evitare che eventi di sicurezza alimentare si verifichino durante le attività produttive.

\subsection{Responsabilità per il tracciamento dei prodotti}

La responsabilità della rintracciabilità degli alimenti all'interno di uno stabilimento di produzione alimentare può variare notevolmente. Può essere complesso individuare una singola entità all'interno dell'organizzazione che sia in grado di gestire tutte le fasi, compreso il ricevimento delle materie prime, la produzione, il dosaggio e la spedizione, e che, al contempo, abbia una piena comprensione dei problemi legati alla sicurezza alimentare che potrebbero richiedere l'attivazione del processo di tracciamento dei prodotti. Per questo motivo, sia l'architettura iniziale e l'istituzione di un sistema di tracciabilità, sia il processo di acquisizione delle informazioni necessarie, dovrebbero coinvolgere un team multidisciplinare.

I membri del team includono le seguenti aree:

\begin{itemize}
    \item \textit{Sicurezza alimentare}: la rintracciabilità diviene di fondamentale importanza in caso di eventi legati alla sicurezza alimentare che richiedono l'identificazione dell'origine dei materiali o, in fasi successive dell'indagine, la comprensione di dove siano stati venduti i materiali potenzialmente contaminati e come siano stati utilizzati. Tuttavia, data la natura stessa della rintracciabilità, il ruolo del professionista della sicurezza alimentare tende a essere periferico e consultivo rispetto al resto del team coinvolto. Il professionista della sicurezza alimentare deve essere competente nei requisiti normativi relativi alla documentazione e deve garantire che il sistema sia progettato con un livello di dettaglio e precisione che tuteli appieno la salute pubblica e, allo stesso tempo, consenta l'individuazione dei prodotti potenzialmente contaminati, minimizzando le perdite per la struttura coinvolta.
    \item \textit{Catena di approvvigionamento e logistica}: la tracciabilità è centrata sulla gestione dei movimenti dei prodotti; di conseguenza, il professionista della catena di approvvigionamento svolge un ruolo cruciale nell'assicurare una tracciabilità efficace.
    \item \textit{Tecnologia informatica}: la tracciabilità è strettamente connessa al movimento dei prodotti. Tuttavia, affinché esista un registro di rintracciabilità, è necessario registrare tale movimento. Sebbene nell'industria alimentare siano ancora diffusi i registri cartacei, c'è una forte preferenza verso l'adozione di registri elettronici. Molte aziende fanno uso di sistemi preesistenti, sviluppati su misura da professionisti per contenere i dati, inclusi quelli essenziali per garantire la rintracciabilità.
    \item \textit{Contabilità}: poiché le merci vengono acquistate e vendute, alcune informazioni rilevanti per la tracciabilità possono essere incorporate nei sistemi contabili. In determinate circostanze, chi effettua l'acquisto o la vendita del prodotto potrebbe non coincidere con chi lo riceve o lo spedisce. Nonostante possa essere utile in certi casi seguire il denaro, è fondamentale tenere presente che la tracciabilità riguarda principalmente il monitoraggio del movimento di oggetti fisici.
    \item \textit{Approvvigionamento e vendite}: coloro che acquistano e vendono prodotti alimentari e mangimi devono essere in grado di avere una chiara visione dell'identità dei propri clienti e di determinare con precisione l'origine e la destinazione delle materie prime e dei prodotti finiti. Inoltre, se il prodotto presenta specifiche indicazioni che richiedono autenticazione, il personale coinvolto nell'approvvigionamento e nelle vendite manifesta un ulteriore interesse nella tracciabilità.
    \item \textit{Spedizione e/o ricezione}: le persone responsabili di gestire i prodotti fisici svolgono un ruolo essenziale nell'assicurare una tracciabilità accurata. Le polizze di carico contengono informazioni cruciali riguardanti il luogo effettivo di spedizione e dovrebbero essere conservate, preferibilmente in formato elettronico, per garantire un registro affidabile.    
    
\end{itemize}

\subsection{Il tracciamento dei prodotti comparato agli altri settori}

Spesso ci si domanda perché la tracciabilità degli alimenti può sembrare molto più complessa rispetto ad altri settori, dove, ad esempio, le compagnie di trasporto merci gestiscono grandi volumi di spedizioni e i clienti possono monitorare la posizione precisa di un pacco online grazie a un codice di tracciamento. Anche i pezzi di ricambio delle automobili sono marcati con informazioni di tracciabilità. La differenza rispetto agli altri settori è dovuta a diversi aspetti che rendono la tracciabilità degli alimenti una sfida più complessa.

Tuttavia, è importante sottolineare che queste sfide non dovrebbero essere utilizzate come giustificazione per rinunciare agli sforzi volti a migliorare la tracciabilità nell'intero sistema alimentare. La disponibilità della tecnologia, combinata a eventuali cambiamenti normativi che potrebbero aumentare la visibilità della catena alimentare, potrebbero consentire un sistema di tracciabilità più veloce ed efficace per i prodotti alimentari.

Come in altri settori, anche gli alimenti viaggiano attraverso il mondo, talvolta attraversando diverse tappe prima di raggiungere il consumatore finale. La natura globale dell'approvvigionamento alimentare non rappresenta di per sé una sfida insormontabile per la tracciabilità. Tuttavia, ci sono diversi fattori che contribuiscono alla complessità della tracciabilità a livello globale.

Uno di questi fattori è la diversità delle normative relative alle informazioni che devono essere associate a un prodotto. Queste normative possono variare da paese a paese e spesso ci sono incoerenze nel modo in cui tali informazioni devono essere formattate e comunicate. Inoltre, la maturità tecnologica varia notevolmente in tutto il mondo, il che può complicare ulteriormente il processo di tracciamento dei prodotti su scala globale.

È importante notare che alcune economie emergenti hanno sfruttato le tecnologie più recenti per semplificare la tracciabilità, mentre altre nazioni, come gli Stati Uniti, che hanno sviluppato le proprie industrie durante la rivoluzione industriale, possono ancora avere impianti di produzione privi delle moderne capacità di telecomunicazione. La sfida, quindi, consiste nel creare un sistema di tracciabilità che possa essere applicato in modo coerente su scala globale, tenendo conto delle diverse normative e delle differenze tecnologiche tra le regioni.

L'approvvigionamento alimentare è un processo globale che coinvolge la distribuzione di alimenti e ingredienti attraverso lunghe distanze geografiche. Indipendentemente dalla distanza fisica percorsa dagli alimenti e dai loro componenti, un prodotto alimentare finito può essere composto da numerosi ingredienti, alcuni dei quali possono averne di propri. Ogni componente può essere stato tagliato, smistato, reimballato e manipolato da diversi membri della catena di approvvigionamento.

In questa intricata rete di produzione e distribuzione, ogni entità lungo il percorso deve acquisire informazioni sulla rintracciabilità e trasmetterle al membro successivo della catena di fornitura. Anche se la registrazione uno a uno sembra semplice quando ci sono solo pochi attori nella catena di fornitura, diventa incredibilmente complessa da gestire quando gli alimenti e gli ingredienti possono essere stati trasferiti e maneggiati decine di volte. La tracciabilità efficace richiede un sistema robusto che sia in grado di gestire questa complessità e garantire che ogni passaggio lungo il percorso sia documentato e rintracciabile in modo accurato.

Il consumo di prodotti alimentari da parte di una famiglia tipica in un anno supera di gran lunga il numero di auto acquistate o dei pacchi consegnati a un determinato indirizzo nello stesso periodo. La produzione alimentare è un processo continuo e i prodotti alimentari hanno un ciclo di vita molto più breve rispetto ad altri beni di consumo.

Le famiglie acquistano generalmente poche auto in un anno, spesso una sola o nessuna, mentre le consegne a un indirizzo specifico possono variare notevolmente ma difficilmente raggiungono la quantità di prodotti alimentari consumati in un anno. La frequenza e l'entità dei consumi alimentari, insieme alla necessità di rifornire costantemente gli scaffali dei negozi, contribuiscono alla complessità e alla rapidità della catena di approvvigionamento alimentare. La tracciabilità degli alimenti è, quindi, fondamentale per garantire la qualità, la sicurezza e la conformità dei prodotti alimentari che consumiamo.

\subsection{Le motivazioni allo sviluppo}

Le applicazioni della rintracciabilità sono estremamente diverse. Quando ci si chiede il perché è importante la rintracciabilità dei prodotti alimentari ci si può aspettare una vasta gamma di risposte tra cui:

\begin{itemize}
    \item la tracciabilità è necessaria per la sicurezza alimentare;
    \item la tracciabilità aiuta a migliorare l'efficienza operativa;
    \item la tracciabilità ci fornisce la visibilità necessaria per comunicare le informazioni richieste dai clienti;
    \item la tracciabilità aiuta ad autenticare le richieste di risarcimento.
    
\end{itemize}

\subsection{Sviluppi futuri}

Ci sono diversi fattori che convergono e che aumentano la probabilità che la tracciabilità degli alimenti continui a migliorare. In primo luogo, i recenti progressi tecnologici e infrastrutturali stanno contribuendo a rendere la tracciabilità degli alimenti più accessibile ed efficace. Sebbene l'adozione e l'implementazione possano essere ancora onerose per alcune aziende e in alcune regioni del mondo, è probabile che nel futuro questi ostacoli tecnologici vengano superati, come spesso avviene con altre applicazioni della tecnologia.

In secondo luogo, c'è un crescente interesse da parte dei consumatori nel conoscere ulteriori informazioni sui prodotti alimentari che acquistano, come la loro origine, gli allergeni, gli ingredienti e altri dettagli.

La crescente consapevolezza della scarsità delle risorse e la necessità di sfamare una popolazione in continua crescita (si prevede che raggiunga quasi dieci miliardi di persone entro il 2050) stanno spingendo l'industria alimentare verso una maggiore efficienza. La tracciabilità offre la visibilità necessaria per identificare le opportunità di miglioramento dell'efficienza in tutto il processo di produzione e distribuzione alimentare, contribuendo a rendere l'industria alimentare più sostenibile e in grado di soddisfare la crescente domanda alimentare globale.

\section{Campi di utilizzo}
\subsection{Salute pubblica}

Il concetto di rintracciabilità degli alimenti in risposta a focolai di malattie di origine alimentare o ad eventi di contaminazione alimentare ha una lunga storia. Già alla fine del 1800 e all'inizio del 1900, gli scienziati avevano stabilito dei collegamenti tra focolai di tubercolosi e streptococco e il consumo di latte crudo. Allo stesso modo, nei primi anni del 1900, i focolai di febbre tifoide sono stati associati al consumo di ostriche.

La rintracciabilità degli alimenti è un elemento fondamentale nella prevenzione e nella gestione di focolai di malattie di origine alimentare e nella garanzia della sicurezza degli alimenti per i consumatori.

All'inizio degli anni '80, numerosi focolai di Norovirus e di epatite A sono stati collegati al consumo di vongole veraci crude o poco cotte. L'inadeguatezza della marcatura dei molluschi ha complicato e rallentato le indagini e la risposta a questi focolai. Più tardi, negli anni '80 e '90, le uova in guscio sono state collegate a numerosi focolai di salmonella. Le indagini su questi focolai sono state difficili perché le informazioni richieste sui contenitori delle uova in guscio indicano dove le uova sono state confezionate, ma non necessariamente dove sono state deposte. A partire dagli anni '90 si sono verificati numerosi focolai di malattie di origine alimentare legati al consumo di prodotti freschi. Le indagini sulla tracciabilità dei prodotti freschi sono state molto difficili da realizzare a causa della complessità del sistema di produzione e distribuzione dei prodotti freschi e dell'inadeguatezza dei registri del sistema. I prodotti freschi di solito non sono accompagnati da informazioni sulla fonte. E quando ciò avviene, le informazioni vengono generalmente scartate dall'utilizzatore finale. Nel 2007, la contaminazione di cibo per animali e latte per l'infanzia con la melamina, motivata da ragioni economiche, ha portato a un'indagine internazionale. La ricerca è stata complicata da un complesso sistema di distribuzione e produzione e da una registrazione inadeguata.

Questi esempi forniscono un contesto cruciale per comprendere perché sono stati investiti tanto tempo ed impegno nell'ottimizzare la rintracciabilità degli alimenti dall'origine fino al punto di consumo. La ragione principale dietro questi sforzi risiede nella stretta relazione tra rintracciabilità e tutela della salute pubblica.

Quando le indagini sui focolai coinvolgono un alimento come veicolo di malattia, gli scienziati devono stabilire l'identità di tale alimento e prevenire ulteriori esposizioni a esso. È anche essenziale determinare il luogo e le circostanze in cui l'alimento è stato contaminato. Nella maggior parte dei casi, i focolai di malattie di origine alimentare segnalati negli Stati Uniti sono il risultato di errori commessi durante la preparazione del cibo, spesso in ristoranti. Tuttavia, se l'indagine rivela che la contaminazione non ha avuto luogo nel punto di preparazione, si avvia una procedura di rintracciabilità per individuare l'origine dell'alimento contaminato. Allo stesso modo, se un campione di prodotto alimentare risulta positivo per la presenza di agenti patogeni anche senza la conferma di casi di malattia, solitamente si avvia un'indagine e una rintracciabilità per prevenire potenziali focolai futuri.

Ogni indagine su un focolaio di malattia alimentare pone gli scienziati di fronte a una sfida delicata, spesso riassunta come il dilemma tra "correttezza e velocità". Questo conflitto è innescato dalla necessità di bilanciare l'urgenza legale ed etica di identificare rapidamente l'alimento contaminato e rimuoverlo dalla circolazione per prevenire ulteriori malattie, con la preoccupazione legale ed etica di non agire troppo rapidamente, identificando l'alimento errato e causando danni economici ingiustificati all'azienda e ai suoi dipendenti.

L'avvertire il pubblico di evitare un alimento che alla fine si scopre non essere stato contaminato non garantisce la protezione, poiché i consumatori potrebbero comunque consumare l'alimento effettivamente contaminato. Un errore del genere può anche minare la credibilità delle future indagini sui focolai presso il pubblico e l'industria. La pressione per agire rapidamente può spingere le agenzie a basarsi su informazioni meno definitive, agendo con una certezza limitata nell'interesse della protezione della salute pubblica. D'altra parte, la necessità di essere corretti può portare a una riluttanza ad agire finché non si dispone di prove più solide, esponendo, così, il pubblico a un maggiore rischio di consumare alimenti contaminati e di sviluppare malattie.

La gestione efficace di questi dilemmi richiede una collaborazione attenta tra le agenzie di salute pubblica, l'industria alimentare e altre parti interessate per garantire una risposta rapida ed efficace ai focolai di malattia alimentare, senza compromettere la precisione delle indagini e la fiducia del pubblico.

\subsubsection{Perché si conducono i tracciamenti}

Le indagini di rintracciabilità rappresentano un tentativo mirato di ricostruire la catena di approvvigionamento alimentare di uno o più alimenti sospettati di essere all'origine di un focolaio di malattia. Quando più persone si ammalano in diverse località nello stesso periodo di tempo e il cibo è identificato come la fonte sospetta, viene avviata un'indagine di rintracciabilità. L'obiettivo principale di questa indagine è individuare un punto di "convergenza" nella catena di approvvigionamento, al fine di identificare la possibile fonte comune di malattia. In altre parole, gli investigatori cercano di risalire al punto in cui tutti i percorsi si incontrano, suggerendo così il possibile luogo o l'origine dell'incidente.

Questo processo di rintracciabilità può essere estremamente complesso, specialmente quando coinvolge una vasta gamma di prodotti alimentari e molteplici fornitori. Tuttavia, è fondamentale per identificare la fonte del problema, isolare i prodotti contaminati e prevenire ulteriori malattie, contribuendo, così, alla protezione della salute pubblica.

Uno degli obiettivi principali di un'indagine di rintracciabilità è quello di prevenire il verificarsi di ulteriori casi di malattia. A tal fine, è necessario che la rintracciabilità sia condotta nel modo più rapido e accurato possibile per individuare una potenziale fonte di contaminazione nella catena di approvvigionamento. La probabilità di interrompere la distribuzione di un prodotto contaminato è minore quando si tratta di prodotti a breve conservazione o ad alta rotazione, come i prodotti freschi, a meno che la fonte di contaminazione non sia continua (acqua di irrigazione, etc.), mentre la probabilità di interrompere la distribuzione di un prodotto a lunga conservazione, come il burro di arachidi, è maggiore.

Identificando la fonte comune dei prodotti contaminati nella catena di approvvigionamento, l'industria e le autorità di regolamentazione possono implementare misure efficaci per prevenire l'ingresso di ulteriori prodotti contaminati sul mercato e rimuovere quelli già contaminati attraverso un richiamo degli alimenti.

All'interno della comunità dei regolatori e della sanità pubblica, si è sviluppata una distinzione sulla base del "tipo" di indagine di rintracciamento che può essere condotta da un'agenzia regolatoria. Quando l'indagine epidemiologica non riesce a identificare in modo definitivo un singolo alimento sospetto associato ai casi di malattia in un focolaio, può essere condotta una rintracciabilità \textit{investigativa} o \textit{epidemiologica} su diversi alimenti per determinare se uno di essi mostra una convergenza nella catena di approvvigionamento che può spiegare la maggior parte delle malattie associate al focolaio. Questo tipo di rintracciamento è condotto con un'attenzione particolare alla rapidità e i dati possono essere raccolti via e-mail o via telefono nel tentativo di ricostruire rapidamente la catena di approvvigionamento e individuare un punto di convergenza.

A differenza di un rintracciamento epidemiologico, un rintracciamento \textit{normativo} viene generalmente avviato quando esiste un forte indicatore che un singolo alimento è associato alle malattie in un focolaio. Nel caso di un rintracciamento normativo, il livello di raccolta delle prove è solitamente più completo: gli investigatori statali negli USA, ad esempio, visitano di persona ogni struttura che ha gestito il prodotto sospetto. Questo tipo di indagine mira a esaminare attentamente la struttura e determinare se ci siano altre fonti di contaminazione plausibili che possano spiegare i casi di malattia nel focolaio, oltre all'alimento sospetto. Ad esempio, se l'alimento sospetto è stato conservato in un magazzino prima di essere distribuito ai negozi di alimentari della zona, gli investigatori cercheranno di determinare se ci sia stata una fonte di contaminazione all'interno del magazzino che potrebbe aver contaminato l'alimento, diventando così la fonte plausibile del focolaio.

\subsubsection{Descrizione di un tracciamento in ambito sanitario}

Un \textit{tracciamento}, nel senso più puro del termine, è semplicemente un'estensione dell'indagine epidemiologica avviata dagli epidemiologi locali o statali e rappresenta uno sforzo per caratterizzare ulteriormente e in modo più accurato l'esposizione a un alimento sospetto per ogni caso di malattia in un focolaio. Di solito, un'indagine di rintracciamento inizia dopo che gli epidemiologi hanno analizzato i casi di malattia in un focolaio e hanno limitato il numero di alimenti sospetti.

A questo punto, gli investigatori delle agenzie statali responsabili della regolamentazione degli alimenti collaborano attivamente con gli epidemiologi per determinare con maggiore precisione i luoghi di esposizione più probabili associati a ciascun caso di focolaio. Questo processo implica anche la raccolta di dati relativi alla distribuzione dei prodotti alimentari nei suddetti luoghi. Ad esempio, una persona potrebbe aver acquistato un alimento contaminato presso un negozio di alimentari, mentre un'altra potrebbe aver consumato lo stesso alimento contaminato in un ristorante situato in una città o stato diverso. Durante un'indagine, ciascuno di questi luoghi in cui si è verificata l'esposizione al cibo sospetto costituisce una "tappa" nell'indagine di tracciabilità, rappresentando un punto chiave in cui uno o più casi sono stati esposti a tale alimento. 

Nel corso delle indagini iniziali in una tappa, le autorità di regolamentazione raccolgono le fatture e le informazioni relative agli ordini di acquisto da ciascuna di queste sedi e iniziano il processo di tracciamento dei prodotti alimentari sospetti attraverso la catena di approvvigionamento. Nella maggior parte dei casi, gli alimenti consegnati e venduti nei punti vendita al dettaglio, come negozi di alimentari o ristoranti, possono essere ricondotti a un'azienda di distribuzione alimentare con un magazzino che può trovarsi nelle vicinanze o anche in stati diversi.

La maggior parte dei produttori e dei distributori di alimenti mantiene una documentazione completa delle transazioni alimentari con altre entità nella catena di approvvigionamento, compresi fornitori e clienti. Queste registrazioni sono state create e rese disponibili alle autorità di regolamentazione coinvolte nelle attività di rintracciamento principalmente perché documentano le transazioni e agevolano la gestione dei pagamenti tra le aziende della filiera alimentare. Gli investigatori di solito si riferiscono a questi documenti, che includono fatture, ordini di acquisto e polizze di carico, per mantenere la tracciabilità \textit{esterna} all'interno della fornitura.

Gli investigatori possono utilizzare questi documenti esterni per tracciare l'alimento sospetto dal magazzino del distributore fino all'installazione di produzione in cui l'alimento è stato creato, considerando il gran numero di ingredienti utilizzati.

Le indagini di rintracciabilità spesso incontrano delle sfide quando si cerca di stabilire la tracciabilità \textit{interna} di un distributore o di un produttore di alimenti. Fino a poco tempo fa, molte aziende operanti nella distribuzione e produzione alimentare non avevano processi o sistemi di gestione dei dati adeguati per mantenere una tracciabilità interna nella catena di approvvigionamento. La tracciabilità interna richiede che i trasformatori o i distributori di alimenti tengano traccia degli elementi interni che possono modificare l'identità o la configurazione del prodotto che vendono. Nel caso della produzione alimentare, la tracciabilità interna può implicare la registrazione e l'archiviazione di tutte le informazioni relative al codice di lotto o alla partita degli ingredienti utilizzati (come cereali, sciroppo di mais, aromi, vitamine, etc.).

Se la tracciabilità interna non viene gestita con precisione e coerenza all'interno di un'azienda o per un determinato alimento oggetto di un'indagine, si verifica una perdita della completa tracciabilità nella catena di approvvigionamento. Questa mancanza di tracciabilità interna è particolarmente comune quando si tratta di alimenti che sono composti da numerosi ingredienti, prodotti sfusi e articoli freschi non confezionati.

Per proseguire con un'indagine di rintracciabilità in queste situazioni, gli investigatori devono stimare la probabilità che un prodotto sospetto sia stato spedito da o ricevuto in una determinata struttura, considerando un intervallo temporale che tenga conto del flusso del prodotto all'interno di quella struttura. Ad esempio, se un prodotto fresco viene generalmente conservato per due giorni in un magazzino refrigerato, l'investigatore potrebbe richiedere la documentazione riguardante le operazioni di ricevimento e spedizione del prodotto per una settimana a partire dalla data di interesse relativa al prodotto alimentare sospetto oggetto di rintracciabilità. Questo intervallo temporale dovrebbe includere tutte le transazioni sospette, sia quelle di ricevimento nel magazzino che quelle di spedizione dal magazzino, al fine di individuare le possibili fonti sospette del prodotto contaminato.

\subsubsection{Le sfide del tracciamento dei prodotti in ambito sanitario}

Se utilizzate come strumento per confermare un'ipotesi di esposizione epidemiologica, le indagini di rintracciamento dovrebbero essere condotte per la maggior parte dei focolai che si sospetta essere collegati ad un alimento distribuito commercialmente. L'applicazione più ampia delle indagini di rintracciamento per identificare le fonti dei focolai dovrebbe portare ad una maggiore scoperta delle fonti rispetto all'uso più limitato di queste indagini. Tuttavia, è importante notare che in entrambi i casi, per ogni indagine di rintracciamento avviata con successo, ve ne sono almeno altrettante che non riescono a individuare una convergenza nella catena di approvvigionamento o a identificare la fonte di un focolaio. Ci sono numerosi motivi per cui un'indagine di rintracciabilità può risultare complessa per un investigatore, sia dal punto di vista normativo che epidemiologico.

La scarsa qualità della documentazione rappresenta spesso un ostacolo rilevante nelle indagini di rintracciabilità che non raggiungono il successo desiderato. Molte aziende alimentari non mantengono i registri necessari per la rintracciabilità in modo adeguato. Questi registri dovrebbero consentire di documentare la provenienza del prodotto e da dove viene spedito in uscita.

Oltre alla mancanza di documenti o alla loro illeggibilità, i prodotti, specialmente quelli sfusi, possono essere identificati con nomi diversi da parte di ciascuna azienda coinvolta nella catena di fornitura. Questa variabilità nella nomenclatura per lo stesso prodotto può generare confusione e ritardi per gli investigatori che cercano di determinare se stanno ancora seguendo la pista del medesimo prodotto.

Anche i prodotti a breve durata di conservazione, come i prodotti freschi e i frutti di mare, possono presentare sfide nella rintracciabilità all'interno della catena di approvvigionamento. Questi articoli spesso non sono confezionati e vengono venduti sfusi o in casse con informazioni limitate sul codice del lotto associato al prodotto alimentare. Inoltre, quando le informazioni sul codice del lotto sono disponibili, la maggior parte dei distributori alimentari non ha la capacità di tracciare dettagliatamente queste informazioni a livello delle singole casse nei propri magazzini. La gestione dei prodotti si basa principalmente sul principio \textit{first-in-first-out}, il che rende difficile agli investigatori determinare i punti vendita in cui il prodotto è stato spedito dal magazzino di distribuzione.

In aggiunta, i prodotti con breve durata di conservazione possono rapidamente esaurirsi dal mercato e talvolta essere completamente consumati dal pubblico prima che un'indagine di rintracciabilità possa essere avviata.

Anche i prodotti alimentari confezionati con una lunga durata di conservazione rappresentano una sfida per gli investigatori poiché le aziende sono tenute a conservare i registri solo per un periodo di 3-4 mesi dopo il completamento di una transazione. Questo diventa problematico quando si tratta di prodotti con una durata di conservazione di 12 mesi, in quanto le informazioni relative alle date di spedizione di tali prodotti possono sfuggire agli investigatori a causa della mancanza di registrazioni disponibili. Sebbene in generale sia più semplice tracciare i prodotti alimentari confezionati rispetto a quelli non confezionati, quando un focolaio è causato da un ingrediente contaminato presente in un prodotto confezionato, come ad esempio la pasta di arachidi utilizzata in un alimento complesso, diversi alimenti confezionati possono essere colpiti. Questo può rendere difficile individuare un punto di convergenza comune nella catena di approvvigionamento, a meno che l'investigatore non riesca a risalire all'origine dell'ingrediente contaminato.

La maggior parte delle aziende alimentari è familiare con il concetto di \textit{richiamo} di un prodotto alimentare, ma spesso c'è una confusione nell'associare questa pratica con un'indagine di rintracciabilità. Nel caso di un richiamo di prodotto, un'azienda semplicemente identifica tutti i luoghi in cui ha distribuito il prodotto interessato, e queste informazioni sono generalmente facilmente reperibili nei propri sistemi di gestione dei dati. In alcuni casi, un'azienda potrebbe essere prudente e richiamare più prodotti di quelli effettivamente potenzialmente contaminati.

Tuttavia, l'indagine di rintracciabilità è un processo molto più specifico e dettagliato rispetto a un semplice richiamo di prodotto. In un'indagine di rintracciabilità, gli investigatori si concentrano su un numero limitato, spesso uno solo, di spedizioni sospette all'interno di un impianto. La specificità richiesta in un'indagine di rintracciabilità supera di gran lunga quella necessaria per un richiamo, e la maggior parte delle aziende incontra difficoltà nel riesaminare in dettaglio tutti i registri pertinenti e nell'identificare un numero limitato di spedizioni sospette all'interno della catena di approvvigionamento.

\subsection{Industria}

Il valore principale dei dati standardizzati risiede nella loro capacità di consentire ai partner commerciali di condividere informazioni e ottenere una visione chiara di ciò che accade lungo la catena di approvvigionamento. La tracciabilità rappresenta un'applicazione fondamentale di questa visibilità della catena di approvvigionamento, poiché utilizza dati basati su eventi per facilitare la rintracciabilità e il ritiro dei prodotti. Tuttavia, è importante notare che la tracciabilità rappresenta solo una delle molteplici applicazioni possibili per queste informazioni.

Le informazioni sulla visibilità della catena di approvvigionamento possono essere sfruttate per migliorare una vasta gamma di processi aziendali. Diverse applicazioni commerciali sfruttano i benefici di una maggiore visibilità della catena di approvvigionamento.

Le implementazioni della visibilità della catena di approvvigionamento forniscono diversi vantaggi a vari settori, come beni di consumo confezionati, vendita al dettaglio/alimentari, servizi di ristorazione, sanità, etc. L'industria alimentare in generale, e la categoria dei prodotti freschi in particolare, hanno un forte interesse a migliorare la visibilità della catena di approvvigionamento a causa di fattori unici del settore, come la natura deperibile dei prodotti, la volatilità dei mercati e la sensibilità alle sfide logistiche impreviste, come condizioni meteorologiche avverse e intoppi nei trasporti. L'implementazione della visibilità della catena di approvvigionamento può avanzare la sicurezza alimentare, rafforzare gli sforzi di sostenibilità e migliorare l'efficienza dei processi aziendali, offrendo all'industria alimentare numerosi vantaggi, tra cui:

\begin{itemize}
    \item efficienza operativa derivante dalla visibilità della catena di fornitura;
    \item miglioramenti sostanziali nei processi aziendali, come la riduzione del lavoro manuale grazie all'ampia adozione del \textit{self-checkout};
    \item una gestione più precisa degli stock, ordini più accurati, maggiore disponibilità dei prodotti e una migliore gestione delle discrepanze nell'inventario;
    \item riduzione dell'impatto economico delle crisi legate alla sicurezza alimentare mediante un rapido isolamento dei prodotti coinvolti e il ripristino della fiducia dei consumatori.
\end{itemize}

Si stima che negli USA questi vantaggi potrebbero equivalere a circa 3 miliardi di dollari soltanto per il settore alimentare fresco, basandosi su un'analisi prudente dei risultati, dei progetti pilota e dei casi di studio settoriali.

\subsubsection{Benefici alla catena di approvvigionamento}

I vantaggi della visibilità della catena di approvvigionamento non si limitano a specifici segmenti di essa. Nel 2013, l'\textit{IFT (Institute of Food Technologies)} ha condotto un progetto pilota in collaborazione con l'industria alimentare per esaminare metodi di tracciabilità alimentare rapidi ed efficienti. Inoltre, ha preparato un dettagliato rapporto chiamato \textit{"Rapporto pilota IFT"} per la \textit{Food and Drug Administration (FDA)} degli Stati Uniti, in conformità con il \textit{Food Safety Modernization Act (FSMA)}. Durante tali progetti pilota sono state condotte interviste ai partecipanti per comprendere i benefici ottenuti dai vari segmenti della catena di approvvigionamento. Oltre ai vantaggi della tracciabilità, i partecipanti ai progetti pilota hanno riportato numerosi altri benefici derivanti dalla visibilità della catena di approvvigionamento, tra cui:

\begin{itemize}
    \item \textit{Distributori}: maggiore capacità di gestione dell'inventario, che consente di fornire informazioni migliori alle forze di vendita, di migliorare i tassi di prelievo e di ridurre le differenze inventariali. Uno dei partecipanti al progetto pilota ha stimato un risparmio combinato sui costi e un aumento delle vendite di 500.000-600.000 dollari.
    \item \textit{Spedizionieri}: visibilità in tempo reale della catena di approvvigionamento sui prodotti effettivamente confezionati sul campo, che ha contribuito a ridurre le situazioni di sovra-vendita o sottovendita giornaliera. Inoltre, la tracciabilità in tempo reale dal campo ai refrigeratori ha migliorato la capacità di assegnare priorità ai carichi destinati ai refrigeratori in base al momento della raccolta dei prodotti.
    \item \textit{Rivenditori}: gestione dell'inventario più efficiente, con una maggiore precisione nelle scorte, un processo di selezione più efficiente e una migliore visibilità della catena di approvvigionamento.
    \item \textit{Produttori}: aumento della produttività, maggiore precisione nell'assicurare che i prodotti corretti siano consegnati ai clienti appropriati e miglior efficienza nelle attività di fatturazione di back-office grazie alla capacità di identificare in modo più agevole i destinatari dei prodotti.
\end{itemize}

\subsubsection{Benefici per la gestione dell'inventario}

Oltre alla tracciabilità, le aziende che implementano la visibilità della catena di fornitura possono sfruttare i dati per migliorare l'efficienza operativa e ottimizzare la gestione dell'inventario, delle risorse e della qualità. Questi miglioramenti nella catena di fornitura possono portare a benefici concreti in termini di profitti. Ad esempio, l'ottimizzazione dei processi di gestione dell'inventario è di vitale importanza per i prodotti deperibili, dove la freschezza e la tempestività delle consegne sono fondamentali per mantenere la massima qualità dei prodotti alimentari.

Inoltre, l'implementazione di processi automatizzati per la visibilità della catena di fornitura può accelerare il movimento di qualsiasi tipo di prodotto attraverso la catena di fornitura, aumentando l'efficienza.

La visibilità della catena di fornitura apporta significativi miglioramenti ai processi di gestione dell'inventario e delle categorie, anche dal punto di vista della domanda. Per quanto riguarda la gestione dell'inventario, l'uso di numeri di lotto e date di ricevimento per identificare in modo univoco i prodotti facilita l'adozione di un approccio \textit{first-in, first-out} per la gestione delle scorte. Questa filosofia può essere sfruttata per impostare avvisi relativi alla rotazione delle scorte, ottimizzando, così, la qualità e minimizzando il deterioramento dei prodotti. Inoltre, consente ai rivenditori di automatizzare in modo più efficiente le riduzioni dei prezzi man mano che le date di scadenza si avvicinano, evitando la vendita di prodotti scaduti. L'identificazione univoca dei prodotti e i dati aggiuntivi consentono anche di programmare automaticamente ribassi dei prezzi e di inviare avvisi sulle date di scadenza, inclusi avvisi relativi a prodotti richiamati.

\subsubsection{Benefici per la brand reputation}

La visibilità della catena di fornitura può contribuire a consolidare la reputazione del marchio e a instillare fiducia nei consumatori. Le informazioni accessibili sulla visibilità possono influenzare le decisioni che hanno un impatto sulla reputazione del marchio, migliorando, quindi, il processo decisionale. Ad esempio, una catena di ristoranti di portata nazionale ha avviato un ambizioso progetto finalizzato a garantire la tracciabilità completa della sua catena di fornitura, al fine di mantenere la promessa del marchio riguardo all'uso degli ingredienti migliori, all'approvvigionamento locale e alle pratiche commerciali sostenibili. L'azienda ha collaborato efficacemente con una vasta rete di fornitori partner per stabilire un sistema di tracciabilità aziendale, consentendo la condivisione di informazioni standardizzate sui prodotti in ogni fase della catena di fornitura.

\subsubsection{Riduzione degli impatti finanziari di un richiamo}

Un efficace sistema di visibilità nella catena di fornitura, associato a un robusto programma di tracciabilità, offre alle aziende l'opportunità di limitare l'impatto economico delle emergenze legate alla sicurezza alimentare, riducendo, così, gli sforzi necessari per ritirare i prodotti dal mercato. I richiami rappresentano un onere significativo per l'industria alimentare fresca, con un costo annuale superiore a 1 miliardo di dollari. 

Un sistema completo di tracciabilità lungo l'intera catena di fornitura può aiutare le aziende a individuare e isolare rapidamente la causa di un'epidemia alimentare, consentendo una risposta tempestiva ed efficace per evitare la rimozione completa dei prodotti potenzialmente contaminati. 

Inoltre, i processi di tracciabilità consentono alle aziende di individuare rapidamente i prodotti sospetti in qualsiasi punto della catena di fornitura, agevolando la rimozione dei prodotti ritirati per proteggere i consumatori e preservare la fiducia dei clienti. Una maggiore visibilità nella catena di approvvigionamento e la consapevolezza dei consumatori sulla tracciabilità degli alimenti freschi dovrebbero anche contribuire a ripristinare la fiducia dei consumatori nella categoria di prodotto coinvolta, accelerando il ritorno alle abitudini di acquisto normali e ai livelli di vendita precedenti. Inoltre, i processi di tracciabilità aiutano le aziende a contenere costi imprevisti (come quelli legali, le multe, il rinnovo forzato dei contratti e la perdita di fedeltà dei clienti) e minimizzano gli impatti negativi sulle aziende coinvolte nella catena di fornitura e sui consumatori.

\subsubsection{Ritorno sull'investimento}

I vantaggi della visibilità nella catena di approvvigionamento influenzano il ritorno sull'investimento (ROI). I miglioramenti dei processi e le efficienze operative che si manifestano in tutta l'organizzazione devono essere considerati quando si valuta il ROI di qualsiasi iniziativa mirata all'implementazione o al miglioramento dei programmi di tracciabilità e visibilità nella catena di fornitura.

Uno strumento che può assistere in questo processo è il \textit{Global Food Traceability Center (GFTC)} il quale aiuta le organizzazioni del settore ittico a valutare l'impatto finanziario, comprendendo costi e benefici, legato all'adozione della tracciabilità.

Il GFTC ha identificato che i benefici dalla tracciabilità possono derivare da:

\begin{itemize}
    \item mercati e clienti nuovi ed ampliati;
    \item riduzione dei costi di responsabilità aziendale;
    \item efficienze di richiamo più elevate e costi di rilavorazione inferiori;
    \item riduzione degli scarti e dei restringimenti del prodotto;
    \item scambio di dati più affidabile e facilmente accessibile con i partner;
    \item costi di adeguamento normativo più rapidi e inferiori.
\end{itemize}

\subsection{Tracciabilità dei prodotti sfusi}

La tracciabilità dei prodotti alimentari sfusi è una sfida unica nel settore alimentare. Le filiere dei prodotti alimentari sfusi possono essere descritte in modo più accurato come un sistema alimentare, a causa della complessa rete che si crea intorno a questi prodotti. Il loro percorso, dalla fattoria o dal fornitore fino al punto finale, non segue un percorso lineare. Spesso, il prodotto viene mescolato con prodotti simili provenienti da diversi lotti o viene completamente unito ad altri prodotti.

I prodotti alimentari sfusi sono particolarmente vulnerabili alla contaminazione incrociata accidentale o alla miscelazione durante lo stoccaggio. Ciò è dovuto sia alla naturale composizione di tali prodotti che al costo e alla praticità di prevenire completamente il mescolamento. Le materie prime vengono raccolte e gli ingredienti o i prodotti sono generalmente conservati in contenitori come serbatoi, bidoni, silos e altri recipienti appropriati. Durante il processo di produzione, i prodotti crudi o parzialmente trasformati possono subire modifiche tramite processi chimici e/o microbiologici, come la pastorizzazione, la sterilizzazione, la concentrazione, la diluizione e la fermentazione. Queste modifiche richiedono la generazione di codici di lotto.

Il monitoraggio di queste informazioni diventa essenziale durante il processo di produzione, che spesso non è continuo e richiede l'immagazzinamento del prodotto intermedio. Anche quando le fasi di lavorazione sono ben definite e i numeri di lotto sono chiaramente identificati, mantenere questa distinzione durante lo stoccaggio può essere difficile.

La tracciabilità dei prodotti alimentari sfusi è definita come la capacità di determinare ogni porzione della composizione di un prodotto grezzo, intermedio o finito in termini di input e output nel sistema. Storicamente, la necessità di identificare queste proporzioni di prodotti alimentari era principalmente guidata dalla gestione dei richiami alimentari. Tuttavia, nel contesto del sistema alimentare statunitense in evoluzione, la trasparenza è emersa come un fattore chiave per le aziende alimentari. L'importanza di mantenere un programma completo di tracciabilità per tutte le linee di prodotto è passata da essere una misura difensiva a diventare una misura opportunistica per migliorare l'efficienza delle aziende alimentari.

Per ottenere una tracciabilità completa dei prodotti sfusi all'interno del sistema alimentare, è necessario registrare tutte le informazioni rilevanti, compresi, ma non limitandosi a numeri di lotto, i pesi, le condizioni di manipolazione e l'elaborazione delle informazioni.

\subsubsection{Gestione della tracciabilità dei prodotti sfusi}

Il processo fondamentale per spostare un singolo prodotto dall'ingresso all'uscita coinvolge numerosi punti di contatto che richiedono la raccolta di dati in ogni fase. Tuttavia, per i prodotti sfusi, questi passaggi intermedi possono rappresentare una sfida maggiore da monitorare. Ciò è dovuto al fatto che, mentre i prodotti vengono mescolati o separati, il produttore deve creare nuove identificazioni o informazioni univoche, come numeri di lotto o altri identificatori, per tracciare i prodotti e i loro componenti avanti e indietro. La gestione di questi passaggi intermedi dipende dall'approccio specifico del produttore alla produzione e alla lavorazione dei propri prodotti.

Esistono due approcci principali alla tracciabilità utilizzati per verificare le affermazioni di sostenibilità e altre pratiche di produzione di prodotti alimentari in varie fasi della catena di approvvigionamento. Questi approcci variano a seconda del grado di controllo richiesto nella gestione dei prodotti. I due schemi di gestione della certificazione includono:

\begin{itemize}
    \item \textit{Segregazione del prodotto}: questo approccio prevede la separazione fisica dei prodotti certificati da quelli non certificati in ogni fase della catena di fornitura.
    \item \textit{Bilancio di massa}: questo approccio consente la presenza di prodotti certificati e non certificati all'interno della catena del valore, a condizione che la percentuale di prodotti certificati superi una soglia specifica (ad esempio, 80\% certificati vs. 20\% non certificati).
   
\end{itemize}

L'applicazione di questi approcci di tracciabilità varia in base alla natura dei prodotti monitorati e ad altri fattori, tra cui i costi e le migliori pratiche. Il costo svolge un ruolo chiave per la maggior parte dei fornitori e delle aziende all'interno della catena del valore, soprattutto quando si tratta di sistemi di bilancio di massa altamente complessi. Maggiore è il controllo richiesto per tracciare un prodotto attraverso il sistema, e maggiore sarà la complessità della tracciabilità, con conseguente aumento dei costi.

\subsubsection{Tecnologie impiegate}

Negli ultimi anni, la ricerca sulla tracciabilità alimentare ha ampliato il proprio campo di interesse, andando oltre la mera sicurezza alimentare e includendo aspetti legati all'efficienza operativa, come la freschezza dei prodotti, la gestione dell'inventario e la riduzione degli sprechi. Tecnologie abilitanti che consentono operazioni più veloci e precise e un'identificazione più economica dei prodotti alimentari sfusi includono dispositivi RFID, sensori connessi in rete, codici a barre lineari e bidimensionali e sistemi diagnostici avanzati. L'utilizzo di questi strumenti in combinazione con dispositivi informatici mobili connessi a Internet consente di identificare, acquisire e condividere rapidamente informazioni su lotti e quantità di prodotto.

Tuttavia, vi sono ancora sfide nella raccolta e condivisione delle informazioni in tutto il sistema. Le materie prime agricole spesso attraversano catene di fornitura lunghe e complesse che si estendono su vaste aree geografiche, attraversano molteplici confini e coinvolgono aziende di varie dimensioni e livelli di tecnologia. Il complesso ambiente normativo, la copertura Internet incompleta e le differenze negli investimenti tecnologici creano un terreno fertile per le sfide nella gestione della tracciabilità degli alimenti sfusi. Nel futuro, ci si può aspettare che la riduzione dei costi, l'aumento dell'efficacia e una maggiore disponibilità delle soluzioni tecnologiche contribuiranno a superare queste sfide. Tuttavia, sarà compito dell'industria alimentare collaborare con i regolatori, gli acquirenti e i programmi di certificazione di terze parti per allineare i requisiti di tracciabilità e garantire un sistema efficace e affidabile.

\subsubsection{Il futuro della tracciabilità dei prodotti sfusi}

In passato, la tracciabilità dei prodotti alimentari sfusi era principalmente influenzata dalle esigenze legate ai cereali e alle materie prime. Tuttavia, il panorama del sistema alimentare sta subendo un rapido cambiamento, non solo negli Stati Uniti ma a livello globale, spingendo tutti gli attori della catena del valore ad adattarsi e a evolversi di conseguenza. Con il mutare delle esigenze e dei desideri dei consumatori, diventa essenziale sviluppare un sistema di tracciamento più accurato e uniforme in tutto il sistema.

Mentre si apportano miglioramenti all'aspetto operativo della tracciabilità, il futuro della tracciabilità dei prodotti sfusi si baserà sull'accelerato sviluppo di tecnologie per la comunicazione e la memorizzazione delle informazioni. L'integrazione di queste tecnologie emergenti sarà fondamentale per soddisfare la crescente necessità di tracciare in modo completo i prodotti alimentari lungo l'intera catena di approvvigionamento. Nonostante le sfide che questo panorama in evoluzione comporta, vi sono anche opportunità di crescita sia nel settore pubblico che in quello privato.

\subsection{Tracciabilità della carne e pollame}
\subsubsection{Il tracciamento degli animali}

La tracciabilità degli animali destinati alla produzione di carne e pollame ha inizio sin dalla fase di identificazione degli stessi. In passato, gli agricoltori segnavano gli animali per scopi di identificazione e di proprietà; nel corso degli ultimi decenni, l'identificazione degli animali è diventata cruciale anche per la tracciabilità in caso di epidemie e malattie animali. Ci sono molte ragioni per monitorare e tracciare gli animali, tra cui la creazione di prodotti a base di carne di valore aggiunto per scopi di marketing, la conformità alle normative, come l'etichettatura del paese di origine e i controlli di biosicurezza, che richiedono una corretta segregazione e separazione degli animali. Pertanto, i programmi di gestione efficiente per aziende agricole e attività di produzione e lavorazione alimentare includono spesso programmi di tracciabilità.

\subsubsection{Tecnologie impiegate}

L'\textit{identificazione a radiofrequenza (RFID)} è una tecnologia disponibile da alcuni decenni, ma il suo utilizzo per scopi di tracciabilità è stato limitato a causa dei costi associati. Nell'industria della carne, ad esempio, la tecnologia RFID viene spesso implementata attraverso l'inserimento di dispositivi nei marchi auricolari dei bovini per tracciare l'identità e la posizione degli animali. L'estrazione delle informazioni dal dispositivo richiede un lettore RFID dedicato, e un sistema software di tracciabilità deve essere configurato per gestire tali informazioni. L'RFID è particolarmente utile perché consente di immagazzinare le informazioni in un dispositivo integrato con limitata interazione umana. Sebbene i codici a barre possano contenere informazioni simili, essi sono molto meno costosi rispetto all'RFID. Attualmente, quest'ultimo è più adatto per prodotti di alto valore o grandi volumi. Tuttavia, è possibile che nel tempo i costi della tecnologia RFID diminuiscano, rendendola più accessibile all'industria.

Stanno emergendo ulteriori tecnologie all'avanguardia per migliorare ulteriormente la tracciabilità dei prodotti animali crudi. Ad esempio, le nuove tecnologie basate sul DNA possono essere utilizzate per consentire ai produttori di carne confezionata, trasformatori e rivenditori alimentari di tracciare l'origine dei prodotti a base di carne fino all'animale di origine. La tracciabilità della carne bovina potrebbe essere notevolmente migliorata prelevando campioni di DNA da ciascuna carcassa prima della trasformazione in carne macinata. Le informazioni relative a ciascun campione verrebbero inserite in un database. Successivamente, il prodotto finito a base di carne macinata potrebbe essere campionato per determinare gli animali specifici utilizzati per produrlo. Queste tecnologie di tracciabilità basate sul DNA potrebbero anche consentire ai partecipanti della catena di fornitura di autenticare e convalidare attributi come la naturalezza, la produzione biologica o la razza Angus dei prodotti a base di carne. Poiché l'identificatore univoco è già presente nella sequenza del DNA, questa tecnologia offre un grande potenziale per il tracciamento di esseri viventi, come gli animali. Le tecniche di sequenziamento e di codifica del DNA stanno diventando sempre più accessibili e possono variare da semplici test di sequenziamento a sequenziamenti completi del genoma, fornendo una vasta gamma di informazioni. Le tecnologie basate sul DNA offrono numerose opportunità nel campo del tracciamento e si prevede che il loro utilizzo continuerà a crescere man mano che la tecnologia migliora.

Infine, per quanto riguarda la distribuzione e la vendita dei prodotti, i codici a barre sono ampiamente utilizzati dall'industria. È possibile applicare etichette con codici a barre su una vasta gamma di prodotti per fornire informazioni sull'origine e sulla distribuzione dei prodotti stessi. Queste tecnologie hanno notevolmente migliorato la tracciabilità dei prodotti durante la distribuzione e la vendita, contribuendo a garantire una maggiore trasparenza nella catena di approvvigionamento.

\section{La blockchain nella tracciabilità dei prodotti}
\subsection{Introduzione alla blockchain}

La storia della tecnologia blockchain ha le sue radici nei settori della tecnologia finanziaria e dell'e-commerce, il che è un aspetto importante da considerare quando si applica questa tecnologia a diversi contesti. Questa tecnologia è emersa inizialmente come un esperimento nel campo dello scambio di valore, noto come \textit{criptovalute}. Il termine \textit{Blockchain} è stato introdotto per la prima volta da \textit{Satoshi Nakamoto}, una figura pseudonima o un gruppo di individui, in un articolo del 2008 che concettualizzava blocchi di dati collegati tramite una catena crittografica in una rete. L'anno successivo, Nakamoto creò \textit{Bitcoin}, basandosi su questo concetto, ed esso rimane ancora oggi la criptovaluta più riconosciuta e diffusa. 

L'obiettivo principale di Nakamoto era quello di creare un sistema in cui le transazioni potessero avvenire senza l'intermediazione di entità consolidate, come le banche, al fine di rendere le transazioni stesse più trasparenti e meno suscettibili di corruzione. L'architettura alla base di queste criptovalute, la blockchain, sfrutta la potenza di una rete globale aperta e la combina con l'uso della crittografia per creare un ambiente sicuro e affidabile che permette lo scambio di valore o informazioni senza bisogno di intermediari di fiducia.

La tecnologia blockchain rappresenta una nuova iterazione di un concetto preesistente; infatti, i registri sono un elemento fondamentale delle attività aziendali, e la blockchain utilizza questa tecnologia per migliorare alcuni dei limiti dei registri tradizionali, riducendo la dipendenza da entità esterne a favore della prova crittografica. Nel contesto della tracciabilità alimentare, le tecnologie blockchain sono considerate uno strumento potente per abilitare una tracciabilità completa e una maggiore trasparenza lungo tutta la catena di approvvigionamento. Con una blockchain distribuita su una rete condivisa, tutte le parti coinvolte nella catena di approvvigionamento possono accedere alle stesse informazioni sulla tracciabilità. Tuttavia, la vera rivoluzione sta nella possibilità che queste informazioni siano disponibili per tutti i segmenti della catena di fornitura, inclusi i consumatori finali.

Sebbene l'inizio della tecnologia blockchain si sia concentrata sulla creazione di criptovalute non istituzionali, la tecnologia è essenzialmente un registro con un vasto potenziale di funzionalità, a seconda dell'architettura. Per il nostro contesto, una transazione è semplicemente l'aggiunta o la modifica di informazioni sulla blockchain. Nella tracciabilità alimentare, ad esempio, un prodotto alimentare può subire una trasformazione interna e sarebbe, quindi, registrato sulla blockchain. Questo processo può essere ancora definito come una transazione, anche se non comporta un trasferimento di denaro.

L'utilizzo della blockchain nei sistemi di tracciabilità si basa su diverse caratteristiche chiave, tra cui la velocità di interrogazione del sistema, la capacità di garantire l'anonimato e la trasparenza simultaneamente, e la natura immutabile e condivisa del sistema. A differenza dei sistemi centralizzati, in cui esiste un singolo punto di vulnerabilità e opacità, la blockchain offre una soluzione che si basa sulla fiducia nella tecnologia anziché nel fornitore. La blockchain potrebbe consentire a diverse parti lungo la catena di fornitura di condividere dati in un registro condiviso che copre l'intera estensione del mercato, dal produttore al consumatore. Inoltre, le aziende possono inserire informazioni sulla tracciabilità senza rivelare dettagli proprietari o strategie di concorrenza.

Dal 2018, le soluzioni di catena di fornitura e tracciabilità basate su blockchain sono state principalmente oggetto di studi pilota e di prime implementazioni. Numerose aziende hanno cominciato ad esplorare l'utilizzo di piattaforme blockchain open source come \textit{Ethereum} o \textit{Hyperledger} di IBM per migliorare le proprie catene di fornitura. Alcuni di questi progetti pilota integrano anche altre tecnologie, come sensori \textit{IoT (Internet of Things)}.

\subsection{Casi d'uso nella tracciabilità}

I casi d'uso della blockchain nella tracciabilità alimentare sono sostanzialmente simili a quelli delle iniziative di tracciabilità in generale, il che spiega il grande interesse di molti leader del settore per questa tecnologia. Le iniziative e le tecnologie di tracciabilità alimentare si concentrano principalmente su cinque principali casi d'uso: la prevenzione delle frodi alimentari, la garanzia della sicurezza alimentare e dei richiami, la conformità normativa, le questioni sociali e la fornitura di informazioni ai consumatori.

La blockchain trova la sua massima utilità nella tracciabilità alimentare quando si tratta di prodotti alimentari, come, ad esempio, prodotti agricoli, e catene di approvvigionamento che sono frammentate, come nel caso dei prodotti ittici. Nei casi in cui le operazioni sono integrate verticalmente, l'utilità delle blockchain è limitata, poiché si possono sfruttare strumenti esistenti nella gestione dell'inventario per raggiungere gli obiettivi di tracciabilità.

Le frodi alimentari rappresentano una sfida per tutti i settori alimentari e richiedono miglioramenti nei metodi di rilevamento e maggiore tracciabilità delle informazioni. La \textit{Food Fraud Initiative} dello Stato del Michigan definisce le frodi alimentari come \textit{"adulterazione, etichettatura errata, manipolazione, superamento o contraffazione, furto, dirottamento, simulazione e contraffazione"}. Spesso, le motivazioni economiche spingono all'alterazione di prodotti alimentari o delle informazioni lungo la catena di fornitura. La blockchain è stata vista come uno strumento potenziale per affrontare le frodi alimentari, poiché fornisce un registro distribuito, inalterabile e datato di ogni fase della catena di approvvigionamento, semplificando così l'audit per indagare su frodi alimentari.

La conformità normativa è un aspetto cruciale nella progettazione dei sistemi di tracciabilità alimentare, poiché la mancanza di conformità può comportare la vendita di prodotti non autorizzati, sanzioni e danni alla reputazione. Con l'aumento dei requisiti normativi per la tracciabilità alimentare su scala globale, la blockchain offre flessibilità ed efficacia nel soddisfare e anticipare tali regolamenti.

Un caso d'uso chiave affrontato direttamente dalla blockchain riguarda i richiami alimentari e la sicurezza. Le epidemie alimentari possono danneggiare gravemente la reputazione delle aziende e la blockchain fornisce un quadro decentralizzato ma unificato per tracciare il percorso dei prodotti alimentari lungo la catena di approvvigionamento. Questo richiede un'architettura dati ben progettata, che sia accessibile e conveniente per tutti i partner della catena di fornitura, promuovendo la condivisione delle responsabilità nella gestione dei dati, l'interoperabilità e la sicurezza dei dati.

La blockchain pubblica Ethereum possiede caratteristiche fondamentali che la rendono un'opzione attraente per il settore della tracciabilità alimentare. In queste blockchain pubbliche, le transazioni possono essere eseguite ad un costo molto ridotto, eliminando, così, gli ostacoli finanziari che spesso affliggono i fornitori, i quali hanno spesso margini di profitto limitati e risorse limitate per l'adozione di nuove tecnologie. Questo facilita notevolmente l'implementazione della tracciabilità.

Uno dei casi d'uso più promettenti nel settore delle startup basate su blockchain è il miglioramento delle informazioni fornite ai consumatori finali sui prodotti alimentari. Il movimento dell'etichettatura pulita e l'industria dei dati dimostrano che i consumatori sono sempre più interessati all'origine, alla produzione e alla catena di approvvigionamento dei prodotti che consumano. L'etichettatura pulita si impegna a garantire che certi ingredienti o additivi non siano stati utilizzati, ma soprattutto rappresenta una tattica di marketing che riflette il desiderio dei consumatori di avere accesso a informazioni dettagliate sui loro prodotti. Le etichette intelligenti rappresentano un altro esempio di come i consumatori desiderino sempre più informazioni. La blockchain, unificando la catena di approvvigionamento, offre alle aziende la possibilità di educare i propri consumatori sull'origine e sulla produzione dei loro prodotti.

\subsection{Configurazioni della blockchain}

I due principali tipi di ambienti blockchain, noti come \textit{blockchain pubblici} e \textit{blockchain privati}, hanno caratteristiche e priorità diverse. Le blockchain pubbliche, come \textit{Bitcoin} ed \textit{Ethereum}, sono aperti a chiunque abbia i token necessari per registrare transazioni sulla blockchain. Tuttavia, il consenso e le configurazioni private sono state sviluppate per adattarsi a esigenze diverse.

Le blockchain hanno priorità concorrenti che determinano l'efficienza e la privacy, e queste priorità variano a seconda del caso d'uso. In una blockchain pubblica veramente decentralizzata, l'accesso è aperto a tutti e le transazioni sono validate attraverso un processo di estrazione basato su token, come avviene con Bitcoin o Ether. Tuttavia, con l'espansione della blockchain, il tempo necessario per completare le transazioni aumenta notevolmente. Questa lentezza potrebbe non essere vantaggiosa per le catene di approvvigionamento, dove l'efficienza è fondamentale, e quindi sono state sviluppate le blockchain private o a consenso. Queste architetture conservano alcune delle caratteristiche desiderabili delle blockchain, come l'immutabilità, la data e l'ora di registrazione e la verificabilità. Tuttavia, rendendo la blockchain privata o semi-privata, si perdono alcune delle caratteristiche innovative della tecnologia, trasformandola essenzialmente in un tipo di database più tradizionale. Questo approccio può aumentare i costi, poiché richiede la gestione di nodi centrali, spesso controllati da un fornitore di servizi, e riduce la democratizzazione e la de-istituzionalizzazione promesse dalla blockchain.

Tuttavia, per le catene di approvvigionamento alimentare, non avere una rete blockchain completamente decentralizzata potrebbe non essere un problema critico. Molte delle attuali implementazioni della blockchain nel settore alimentare sono fornite da grandi aziende verticalmente integrate, che hanno le risorse per negoziare con i fornitori di servizi per coordinare, ospitare e gestire nodi di fiducia nella rete. Queste aziende possono anche collaborare direttamente con i partner della catena di fornitura per garantire l'adozione efficace delle migliori pratiche e tecnologie.
